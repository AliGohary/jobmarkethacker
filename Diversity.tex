\documentclass[11pt,letterpaper]{article}
\usepackage[margin=1.00in]{geometry}
\usepackage{setspace}
\usepackage{placeins}
\usepackage[english]{babel}
\usepackage[utf8]{inputenc}
\usepackage{fancyhdr}
\usepackage{hyperref}
\usepackage{import}
\usepackage{datatool}
\usepackage{ifthen}
\usepackage{pxfonts}
\usepackage{blindtext} % Only used to fill space

 

\fancypagestyle{style1}{
\fancyhf{}
\rhead{Diversity Statement\\ November 2019}
\lhead{Daniel Mangrum\\ Vanderbilt University}
%\rfoot{Page \thepage}
}

\fancypagestyle{style2}{
\fancyhf{}
\rhead{Diversity Statement\\ page \thepage}
\lhead{Daniel Mangrum\\ Vanderbilt University}
%\rfoot{Page \thepage}
}


\begin{document}


\DTLloaddb{sample}{Jobs_export.csv}
\DTLforeach{sample}{%
\Complete=Complete,%
\DiversityStatement=DiversityStatement,%
\Submitted=Submitted,%
\DiversityBonus=DiversityBonus%
}{
	\ifthenelse{\equal{\DiversityStatement}{} \OR \equal{\Submitted}{Yes}}{}{
		\ifthenelse{\equal{\Complete}{Yes}}{
			\setcounter{page}{1}
			\pagestyle{style1}

			% I only had one version of my diversity and inclusion statement and used the DiversityBonus wildcard in the Job_details csv to customize it when needed. You can create multiple statements using a similar template as the teaching statement if you'd like.
	 		\ifthenelse{\equal{\DiversityStatement}{standard}}{

			\Blindtext[6]% Only used to fill space


			\bigskip
			\DiversityBonus


			\clearpage
			}{}
		}{}
	}
}

\end{document}

